% Options for packages loaded elsewhere
% Options for packages loaded elsewhere
\PassOptionsToPackage{unicode}{hyperref}
\PassOptionsToPackage{hyphens}{url}
\PassOptionsToPackage{dvipsnames,svgnames,x11names}{xcolor}
%
\documentclass[
  letterpaper,
  DIV=11,
  numbers=noendperiod]{scrartcl}
\usepackage{xcolor}
\usepackage{amsmath,amssymb}
\setcounter{secnumdepth}{5}
\usepackage{iftex}
\ifPDFTeX
  \usepackage[T1]{fontenc}
  \usepackage[utf8]{inputenc}
  \usepackage{textcomp} % provide euro and other symbols
\else % if luatex or xetex
  \usepackage{unicode-math} % this also loads fontspec
  \defaultfontfeatures{Scale=MatchLowercase}
  \defaultfontfeatures[\rmfamily]{Ligatures=TeX,Scale=1}
\fi
\usepackage{lmodern}
\ifPDFTeX\else
  % xetex/luatex font selection
\fi
% Use upquote if available, for straight quotes in verbatim environments
\IfFileExists{upquote.sty}{\usepackage{upquote}}{}
\IfFileExists{microtype.sty}{% use microtype if available
  \usepackage[]{microtype}
  \UseMicrotypeSet[protrusion]{basicmath} % disable protrusion for tt fonts
}{}
\makeatletter
\@ifundefined{KOMAClassName}{% if non-KOMA class
  \IfFileExists{parskip.sty}{%
    \usepackage{parskip}
  }{% else
    \setlength{\parindent}{0pt}
    \setlength{\parskip}{6pt plus 2pt minus 1pt}}
}{% if KOMA class
  \KOMAoptions{parskip=half}}
\makeatother
% Make \paragraph and \subparagraph free-standing
\makeatletter
\ifx\paragraph\undefined\else
  \let\oldparagraph\paragraph
  \renewcommand{\paragraph}{
    \@ifstar
      \xxxParagraphStar
      \xxxParagraphNoStar
  }
  \newcommand{\xxxParagraphStar}[1]{\oldparagraph*{#1}\mbox{}}
  \newcommand{\xxxParagraphNoStar}[1]{\oldparagraph{#1}\mbox{}}
\fi
\ifx\subparagraph\undefined\else
  \let\oldsubparagraph\subparagraph
  \renewcommand{\subparagraph}{
    \@ifstar
      \xxxSubParagraphStar
      \xxxSubParagraphNoStar
  }
  \newcommand{\xxxSubParagraphStar}[1]{\oldsubparagraph*{#1}\mbox{}}
  \newcommand{\xxxSubParagraphNoStar}[1]{\oldsubparagraph{#1}\mbox{}}
\fi
\makeatother

\usepackage{color}
\usepackage{fancyvrb}
\newcommand{\VerbBar}{|}
\newcommand{\VERB}{\Verb[commandchars=\\\{\}]}
\DefineVerbatimEnvironment{Highlighting}{Verbatim}{commandchars=\\\{\}}
% Add ',fontsize=\small' for more characters per line
\usepackage{framed}
\definecolor{shadecolor}{RGB}{241,243,245}
\newenvironment{Shaded}{\begin{snugshade}}{\end{snugshade}}
\newcommand{\AlertTok}[1]{\textcolor[rgb]{0.68,0.00,0.00}{#1}}
\newcommand{\AnnotationTok}[1]{\textcolor[rgb]{0.37,0.37,0.37}{#1}}
\newcommand{\AttributeTok}[1]{\textcolor[rgb]{0.40,0.45,0.13}{#1}}
\newcommand{\BaseNTok}[1]{\textcolor[rgb]{0.68,0.00,0.00}{#1}}
\newcommand{\BuiltInTok}[1]{\textcolor[rgb]{0.00,0.23,0.31}{#1}}
\newcommand{\CharTok}[1]{\textcolor[rgb]{0.13,0.47,0.30}{#1}}
\newcommand{\CommentTok}[1]{\textcolor[rgb]{0.37,0.37,0.37}{#1}}
\newcommand{\CommentVarTok}[1]{\textcolor[rgb]{0.37,0.37,0.37}{\textit{#1}}}
\newcommand{\ConstantTok}[1]{\textcolor[rgb]{0.56,0.35,0.01}{#1}}
\newcommand{\ControlFlowTok}[1]{\textcolor[rgb]{0.00,0.23,0.31}{\textbf{#1}}}
\newcommand{\DataTypeTok}[1]{\textcolor[rgb]{0.68,0.00,0.00}{#1}}
\newcommand{\DecValTok}[1]{\textcolor[rgb]{0.68,0.00,0.00}{#1}}
\newcommand{\DocumentationTok}[1]{\textcolor[rgb]{0.37,0.37,0.37}{\textit{#1}}}
\newcommand{\ErrorTok}[1]{\textcolor[rgb]{0.68,0.00,0.00}{#1}}
\newcommand{\ExtensionTok}[1]{\textcolor[rgb]{0.00,0.23,0.31}{#1}}
\newcommand{\FloatTok}[1]{\textcolor[rgb]{0.68,0.00,0.00}{#1}}
\newcommand{\FunctionTok}[1]{\textcolor[rgb]{0.28,0.35,0.67}{#1}}
\newcommand{\ImportTok}[1]{\textcolor[rgb]{0.00,0.46,0.62}{#1}}
\newcommand{\InformationTok}[1]{\textcolor[rgb]{0.37,0.37,0.37}{#1}}
\newcommand{\KeywordTok}[1]{\textcolor[rgb]{0.00,0.23,0.31}{\textbf{#1}}}
\newcommand{\NormalTok}[1]{\textcolor[rgb]{0.00,0.23,0.31}{#1}}
\newcommand{\OperatorTok}[1]{\textcolor[rgb]{0.37,0.37,0.37}{#1}}
\newcommand{\OtherTok}[1]{\textcolor[rgb]{0.00,0.23,0.31}{#1}}
\newcommand{\PreprocessorTok}[1]{\textcolor[rgb]{0.68,0.00,0.00}{#1}}
\newcommand{\RegionMarkerTok}[1]{\textcolor[rgb]{0.00,0.23,0.31}{#1}}
\newcommand{\SpecialCharTok}[1]{\textcolor[rgb]{0.37,0.37,0.37}{#1}}
\newcommand{\SpecialStringTok}[1]{\textcolor[rgb]{0.13,0.47,0.30}{#1}}
\newcommand{\StringTok}[1]{\textcolor[rgb]{0.13,0.47,0.30}{#1}}
\newcommand{\VariableTok}[1]{\textcolor[rgb]{0.07,0.07,0.07}{#1}}
\newcommand{\VerbatimStringTok}[1]{\textcolor[rgb]{0.13,0.47,0.30}{#1}}
\newcommand{\WarningTok}[1]{\textcolor[rgb]{0.37,0.37,0.37}{\textit{#1}}}

\usepackage{longtable,booktabs,array}
\usepackage{calc} % for calculating minipage widths
% Correct order of tables after \paragraph or \subparagraph
\usepackage{etoolbox}
\makeatletter
\patchcmd\longtable{\par}{\if@noskipsec\mbox{}\fi\par}{}{}
\makeatother
% Allow footnotes in longtable head/foot
\IfFileExists{footnotehyper.sty}{\usepackage{footnotehyper}}{\usepackage{footnote}}
\makesavenoteenv{longtable}
\usepackage{graphicx}
\makeatletter
\newsavebox\pandoc@box
\newcommand*\pandocbounded[1]{% scales image to fit in text height/width
  \sbox\pandoc@box{#1}%
  \Gscale@div\@tempa{\textheight}{\dimexpr\ht\pandoc@box+\dp\pandoc@box\relax}%
  \Gscale@div\@tempb{\linewidth}{\wd\pandoc@box}%
  \ifdim\@tempb\p@<\@tempa\p@\let\@tempa\@tempb\fi% select the smaller of both
  \ifdim\@tempa\p@<\p@\scalebox{\@tempa}{\usebox\pandoc@box}%
  \else\usebox{\pandoc@box}%
  \fi%
}
% Set default figure placement to htbp
\def\fps@figure{htbp}
\makeatother





\setlength{\emergencystretch}{3em} % prevent overfull lines

\providecommand{\tightlist}{%
  \setlength{\itemsep}{0pt}\setlength{\parskip}{0pt}}



 


\KOMAoption{captions}{tableheading}
\makeatletter
\@ifpackageloaded{caption}{}{\usepackage{caption}}
\AtBeginDocument{%
\ifdefined\contentsname
  \renewcommand*\contentsname{Table of contents}
\else
  \newcommand\contentsname{Table of contents}
\fi
\ifdefined\listfigurename
  \renewcommand*\listfigurename{List of Figures}
\else
  \newcommand\listfigurename{List of Figures}
\fi
\ifdefined\listtablename
  \renewcommand*\listtablename{List of Tables}
\else
  \newcommand\listtablename{List of Tables}
\fi
\ifdefined\figurename
  \renewcommand*\figurename{Figure}
\else
  \newcommand\figurename{Figure}
\fi
\ifdefined\tablename
  \renewcommand*\tablename{Table}
\else
  \newcommand\tablename{Table}
\fi
}
\@ifpackageloaded{float}{}{\usepackage{float}}
\floatstyle{ruled}
\@ifundefined{c@chapter}{\newfloat{codelisting}{h}{lop}}{\newfloat{codelisting}{h}{lop}[chapter]}
\floatname{codelisting}{Listing}
\newcommand*\listoflistings{\listof{codelisting}{List of Listings}}
\makeatother
\makeatletter
\makeatother
\makeatletter
\@ifpackageloaded{caption}{}{\usepackage{caption}}
\@ifpackageloaded{subcaption}{}{\usepackage{subcaption}}
\makeatother
\usepackage{bookmark}
\IfFileExists{xurl.sty}{\usepackage{xurl}}{} % add URL line breaks if available
\urlstyle{same}
\hypersetup{
  pdftitle={DATA210P HW2 - Bike Sharing (hour.csv): Linear Modeling, Selection, Validation, and Ridge \& Lasso},
  pdfauthor={Joe Nguyen, Haesung Becker, Jared Lyon, Tao Chen},
  colorlinks=true,
  linkcolor={blue},
  filecolor={Maroon},
  citecolor={Blue},
  urlcolor={Blue},
  pdfcreator={LaTeX via pandoc}}


\title{DATA210P HW2 - Bike Sharing (hour.csv): Linear Modeling,
Selection, Validation, and Ridge \& Lasso}
\author{Joe Nguyen, Haesung Becker, Jared Lyon, Tao Chen}
\date{}
\begin{document}
\maketitle

\renewcommand*\contentsname{Table of contents}
{
\hypersetup{linkcolor=}
\setcounter{tocdepth}{3}
\tableofcontents
}

Load libraries \& packages

\begin{Shaded}
\begin{Highlighting}[]
\ImportTok{import}\NormalTok{ pandas }\ImportTok{as}\NormalTok{ pd}
\ImportTok{import}\NormalTok{ numpy }\ImportTok{as}\NormalTok{ np}

\ImportTok{import}\NormalTok{ matplotlib.pyplot }\ImportTok{as}\NormalTok{ plt}
\ImportTok{import}\NormalTok{ seaborn }\ImportTok{as}\NormalTok{ sns}
\ImportTok{from}\NormalTok{ textwrap }\ImportTok{import}\NormalTok{ dedent}
\end{Highlighting}
\end{Shaded}

Import UCI ML Repo and load dataset (hour.csv). We decided to use the
hour.csv dataset for this homework assignment because it contains more
data points for a more robust analysis.

\begin{Shaded}
\begin{Highlighting}[]
\ImportTok{from}\NormalTok{ ucimlrepo }\ImportTok{import}\NormalTok{ fetch\_ucirepo }
  
\CommentTok{\# fetch dataset }
\NormalTok{bike\_sharing }\OperatorTok{=}\NormalTok{ fetch\_ucirepo(}\BuiltInTok{id}\OperatorTok{=}\DecValTok{275}\NormalTok{) }
  
\CommentTok{\# data (as pandas dataframes) }
\NormalTok{X }\OperatorTok{=}\NormalTok{ bike\_sharing.data.features }
\NormalTok{y }\OperatorTok{=}\NormalTok{ bike\_sharing.data.targets }
  
\CommentTok{\# metadata }
\CommentTok{\# print(bike\_sharing.metadata) }
  
\CommentTok{\# variable information }
\CommentTok{\# print(bike\_sharing.variables) }
\end{Highlighting}
\end{Shaded}

Initialize dataframe and perform initial data exploration

\begin{Shaded}
\begin{Highlighting}[]
\NormalTok{df }\OperatorTok{=}\NormalTok{ X.copy()}
\NormalTok{df }\OperatorTok{=}\NormalTok{ df.join(y)}

\NormalTok{df.shape, df.columns[:}\DecValTok{14}\NormalTok{]}

\NormalTok{wide }\OperatorTok{=}\NormalTok{ df.head()}

\NormalTok{latex }\OperatorTok{=}\NormalTok{ wide.to\_latex(index}\OperatorTok{=}\VariableTok{False}\NormalTok{, escape}\OperatorTok{=}\VariableTok{True}\NormalTok{)}
\BuiltInTok{print}\NormalTok{(}\VerbatimStringTok{r"}\DecValTok{\textbackslash{}b}\VerbatimStringTok{egin\{table\}}\PreprocessorTok{[H]}\VerbatimStringTok{"}\NormalTok{)}
\BuiltInTok{print}\NormalTok{(}\VerbatimStringTok{r"}\ErrorTok{\textbackslash{}}\VerbatimStringTok{centering"}\NormalTok{)}
\BuiltInTok{print}\NormalTok{(}\VerbatimStringTok{r"}\CharTok{\textbackslash{}r}\VerbatimStringTok{esizebox\{}\CharTok{\textbackslash{}t}\VerbatimStringTok{extwidth\}\{!\}\{\%"}\NormalTok{)}
\BuiltInTok{print}\NormalTok{(latex)}
\BuiltInTok{print}\NormalTok{(}\VerbatimStringTok{r"\}"}\NormalTok{)}
\BuiltInTok{print}\NormalTok{(}\VerbatimStringTok{r"}\ErrorTok{\textbackslash{}}\VerbatimStringTok{caption\{First rows of the dataset}\DecValTok{.}\VerbatimStringTok{\}"}\NormalTok{)}
\BuiltInTok{print}\NormalTok{(}\VerbatimStringTok{r"}\ErrorTok{\textbackslash{}}\VerbatimStringTok{label\{tab:first5{-}hourly\}"}\NormalTok{)}
\BuiltInTok{print}\NormalTok{(}\VerbatimStringTok{r"}\ErrorTok{\textbackslash{}}\VerbatimStringTok{end\{table\}"}\NormalTok{)}
\end{Highlighting}
\end{Shaded}

\begin{table}[H]
\centering
\resizebox{\textwidth}{!}{%
\begin{tabular}{lrrrrrrrrrrrrr}
\toprule
dteday & season & yr & mnth & hr & holiday & weekday & workingday & weathersit & temp & atemp & hum & windspeed & cnt \\
\midrule
2011-01-01 & 1 & 0 & 1 & 0 & 0 & 6 & 0 & 1 & 0.240000 & 0.287900 & 0.810000 & 0.000000 & 16 \\
2011-01-01 & 1 & 0 & 1 & 1 & 0 & 6 & 0 & 1 & 0.220000 & 0.272700 & 0.800000 & 0.000000 & 40 \\
2011-01-01 & 1 & 0 & 1 & 2 & 0 & 6 & 0 & 1 & 0.220000 & 0.272700 & 0.800000 & 0.000000 & 32 \\
2011-01-01 & 1 & 0 & 1 & 3 & 0 & 6 & 0 & 1 & 0.240000 & 0.287900 & 0.750000 & 0.000000 & 13 \\
2011-01-01 & 1 & 0 & 1 & 4 & 0 & 6 & 0 & 1 & 0.240000 & 0.287900 & 0.750000 & 0.000000 & 1 \\
\bottomrule
\end{tabular}

}
\caption{First rows of the dataset.}
\label{tab:first5-hourly}
\end{table}

\section{Part 1: Linear Model and
Interpretation}\label{part-1-linear-model-and-interpretation}

\begin{Shaded}
\begin{Highlighting}[]
\NormalTok{target }\OperatorTok{=}\NormalTok{ y.columns[}\DecValTok{0}\NormalTok{]}

\NormalTok{plt.figure()}
\NormalTok{plt.hist(df[target], bins}\OperatorTok{=}\DecValTok{30}\NormalTok{)}
\NormalTok{plt.xlabel(target)}
\NormalTok{plt.ylabel(}\StringTok{"Frequency"}\NormalTok{)}
\NormalTok{plt.title(}\StringTok{"Bike rentals distribution"}\NormalTok{)}
\NormalTok{plt.show()}
\end{Highlighting}
\end{Shaded}

\begin{figure}[H]

\centering{

\pandocbounded{\includegraphics[keepaspectratio]{02_bike_sharing_files/figure-pdf/fig-cnt-hist-output-1.png}}

}

\caption{\label{fig-cnt-hist}Distribution of total bike rentals (cnt).}

\end{figure}%




\end{document}
